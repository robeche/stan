\chapter{Conclusiones y Trabajo Futuro}
%================================================

\section{Logros del Proyecto}

\begin{itemize}
    \item [\checkmark] Pipeline RAG completo funcional
    \item [\checkmark] Procesamiento automatico de documentos
    \item [\checkmark] Busqueda vectorial eficiente
    \item [\checkmark] Chatbot conversacional multilingue
    \item [\checkmark] Integracion de tablas e imagenes
    \item [\checkmark] Interface administrativa completa
    \item [\checkmark] Procesamiento asincrono robusto
\end{itemize}

\section{Trabajo Futuro}

\subsection{Funcionalidades}

\begin{itemize}
    \item Reranking de resultados
    \item Multi-query fusion
    \item Feedback de usuarios sobre respuestas
    \item Exportar conversaciones a PDF
    \item API REST para integración externa
    \item Autenticación de usuarios para chatbot
    \item Dashboard de analytics
\end{itemize}

\subsection{Optimizaciones}

\begin{itemize}
    \item Quantización de modelos
    \item Cacheo de embeddings frecuentes
    \item Streaming de respuestas LLM
    \item Pre-fetching de imágenes
    \item Compresión de imágenes
\end{itemize}

\subsection{Escalabilidad}

\begin{itemize}
    \item Migración a PostgreSQL
    \item Implementar load balancing
    \item Kubernetes deployment
    \item Multi-tenancy support
    \item Distributed ChromaDB
\end{itemize}

\section{Reflexión Final}

Este proyecto demuestra la viabilidad de implementar un sistema RAG completo utilizando tecnologías open-source y modelos de última generación. La combinación de Django, ChromaDB, y Ollama proporciona una base sólida para aplicaciones de IA conversacional basadas en documentos.

La arquitectura modular permite fácil extensión y personalización, mientras que el procesamiento asíncrono garantiza una experiencia de usuario fluida incluso con documentos grandes.

%================================================

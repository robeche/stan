\chapter{Introducción}
%================================================

\section{Resumen Ejecutivo}

Este proyecto implementa un sistema completo de \textbf{Retrieval-Augmented Generation (RAG)} utilizando Django como framework web. El sistema permite procesar documentos técnicos complejos, extraer información estructurada, y proporcionar una interfaz de chatbot inteligente para consultar la información mediante procesamiento de lenguaje natural.

\subsection{Características Principales}

\begin{itemize}
    \item \textbf{Procesamiento de documentos}: Parsing automático con Nemotron Parse v1.1
    \item \textbf{Extracción inteligente}: Imágenes, tablas, y texto estructurado
    \item \textbf{Chunking semántico}: División contextual del contenido
    \item \textbf{Embeddings de alto rendimiento}: BGE-M3 con GPU (CUDA)
    \item \textbf{Búsqueda vectorial}: ChromaDB para retrieval eficiente
    \item \textbf{Chatbot conversacional}: Respuestas basadas en contexto con Ollama
    \item \textbf{Procesamiento asíncrono}: Celery + Redis para tareas en background
    \item \textbf{Interfaz administrativa}: Panel completo de gestión
\end{itemize}

\section{Objetivos del Sistema}

\begin{enumerate}
    \item Automatizar el procesamiento de documentos técnicos complejos
    \item Proporcionar acceso rápido a información mediante búsqueda semántica
    \item Generar respuestas contextuales en lenguaje natural
    \item Mantener trazabilidad de fuentes y referencias
    \item Soportar múltiples idiomas de forma transparente
\end{enumerate}


